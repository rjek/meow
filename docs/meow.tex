\documentclass[a4paper,twoside,openany]{book}

\author{\copyright~\href{http://www.rjek.com/}{Rob Kendrick} and
         \href{http://www.digital-scurf.org}{Daniel Silverstone}, 2007}
\title{MEOW MCU Reference Guide, Version 0}
\date{02 March 2007}

\pagestyle{headings}
\usepackage{bytefield}

% let's make this document not look repugnant
\usepackage{a4wide}
\usepackage{palatino}

\usepackage[pdftex]{hyperref} % must be last thing in document preamble.

\begin{document}
  \tableofcontents{}
  
  \chapter{Introduction}
    MEOW is a trivial ARM-inspired 32 bit RISC microcontroller, with 16 bit
    instructions.  It features a simple memory map, basic multi-processing,
    and an easy-to-pick-up instruction set.

    MEOW is a disingenuous acronym for Microprocessor With Eight Opcodes,
    which demonstrates the MCU's simplicity.  The design is kept this simple
    such that it is easy to fit in small, hobbyist, cheap FPGAs, as the
    initial purpose of the project was to build a computer from scratch,
    including the CPU and system controller.

    People familiar with the ARM should be comfortable with the MEOW
    instruction set, as it is designed to be similar.  It is 16 bits wide,
    much like Thumb, however only the Branch instruction is conditionally
    executable.
    
    This reference guide describes \emph{MEOW v0}, the initial version of
    the architecture.  It also describes its system controller, Chairman,
    which provides several built-in devices as well as managing connected
    devices, and the mail-box system.  We also discuss the suggested ABI,
    which is used by our own C compiler.
  
  \chapter{Registers}
    MEOW uses 4 bits to describe registers in its instructions, and thus has
    16 available, each being 32 bits wide and suitable for holding either
    signed, non-signed, or arbitary data.
    
    Most of these registers are completely general-purpose, where some are 
    special-purpose.  Others inhabit the range between the two.  Registers are
    usually referred to using \emph{Rx} notation, such that the first
    register is \emph{R0}, the second \emph{R1} and so on.  There are also
    aliases for the special-purpose registers, such that symbolic names
    such as \emph{pc} can be used in place of \emph{R15}.
    
    \emph{R0 $\to$ R12} are completely general-purpose.  The CPU has no
    specific use for them, and the programmer is free to use them entirely as
    they wish.  Any operating system, calling standard or ABI may of course
    specify uses for them however.  We discuss a suggested ABI, MABI towards
    the end of this document.
    
    \emph{R11 $\to$ R15} vary between recommended usage, and specific purpose.
    \begin{figure}[h]
      \begin{tabular}{c c l}
        Register   & Alias     & Purpose\\
        \hline
        \emph{R0 $\to$ R10} &  & General purpose\\
        \emph{R11} & \emph{sp} & Recommended as stack pointer, general purpose\\
        \emph{R12} & \emph{lr} & Recommended as link register, general purpose\\
        \emph{R13} & \emph{ir} & Immediate register, recommended as\\
                   &           & syscall register, general purpose\\
        \emph{R14} & \emph{sr} & Status register\\
        \emph{R15} & \emph{pc} & Program counter
      \end{tabular}
      \caption{Register usage and aliases}
      \label{fig:registers}
    \end{figure}

    \section{Register banks}
      MEOW has two banks of registers, and swaps between them when entering
      interrupt mode, such that it can have its own stack, and other
      environmental values untouched by the program executing.  Unlike some
      other CPUs, all registers are banked, meaning that MEOW has 32 registers
      in total.
      
      Certain instructions can act on registers within the other bank, such as
      CMP and MOV.  We refer to the registers in the other bank as the
      ``alternative'' registers or bank, and use \emph{ARx} to describe them
      in source code when we wish to access the register in the alternative
      bank rather than the current.  An assembler may extend its aliases for
      special-purpose registers by prefixing an A also, such that the
      alternative bank's program counter may be refered to as \emph{apc}.

    \section{Special-purpose registers}
      \subsection{Status Register}
        \label{sec:statusregister}
        The status register contains flags describing the current state of
        the MCU.  These include the condition flags used for deciding if a
        branch should be taken, and if we are in interrupt mode.
	
	\setlength{\bitwidth}{1em}

        \begin{figure}[h]
          \begin{bytefield}{32}
            \bitheader[b]{0, 1, 27, 28, 29, 30, 31}\\
            \bitbox{1}{N} &
            \bitbox{1}{Z} &
            \bitbox{1}{C} &
            \bitbox{1}{V} &
            \bitbox{27}{undefined} &
            \bitbox{1}{I}\\
          \end{bytefield}
          
          \begin{tabular}{r l}
            Bit & Meaning\\
            \hline
            N & Non-zero\\
            Z & Zero\\
            C & Carry\\
            V & Overflow\\
            Undefined & No current meaning. Maintain the values stored here.\\
            I & Set if in interrupt mode.  \emph{Read-only}.
          \end{tabular}
          \caption{Contents of the status register}
          \label{fig:statusregister}
        \end{figure}
        

      \subsection{Program Counter}
        The program counter register contains the address in memory of the
        currently executing instruction.  As MEOW is not a pipelined CPU,
        there are no surprises here --- instructions can reference themselves
        via \emph{pc} without applying any offset.
        \begin{figure}[h]
          \begin{bytefield}{32}
            \bitheader[b]{0, 1, 31}\\
            \bitbox{31}{address} &
            \bitbox{1}{X}\\
          \end{bytefield}\\
          MEOW only supports executing instructions that are on 2-byte
          boundries, and as such \emph{X} must always be zero.  What happens
          when it is set to one is \emph{UNDEFINED}.
          \caption{Contents of the program counter}
          \label{fig:programcounter}
        \end{figure}        

  \chapter{Instructions}
    \setlength{\bitwidth}{2em} % All the bitfields in this chapter are 16 bits,
                               % and they look a bit puny compared to the 32
                               % bit ones earlier, so we make them a bit wider

    This chapter describes each of MEOW's instructions, including their bit
    encodings, and suggested assembler mnemonics.
    
    All of MEOW's instructions are 16 bits wide, and have their opcode encoded
    into the first three bits.  The other bits of the instructions are
    reasonably consistently used.  For example, bits 11 $\to$ 8 are always used
    to store the destination register if the instruction acts on one.
    
    While some of the instructions will seem familiar to ARM programmers,
    others such as the \texttt{BIT} and \texttt{MEM} instructions will not.
    However, assembler macros are usually built using these instructions in
    order to provide more friendly ARM-like instructions such as \texttt{AND},
    \texttt{EOR}, \texttt{LDR} etc.
    
    \newpage
    
    \section{B --- Branch}
      \begin{center}\begin{bytefield}{16}
        \bitheader[b]{0, 8, 9, 12, 13, 14, 15}\\
        \bitbox[ltb]{1}{0} & \bitbox[tb]{1}{0} & \bitbox[tb]{1}{0} & 
	\bitbox{4}{cond} &
	\bitbox{9}{signed offset}
      \end{bytefield}\end{center}
      
      Tells the MCU to start executing instructions from another location in
      memory, if the conditions are right.
      \subsection*{Syntax}
        \texttt{B\{cond\} \#offset}
        
        The condition part of the instruction is optional, and if it is
        omitted, it is assumed that it is \emph{AL}.  The offset is a signed
        9 bit immediate, shifted right one bit so that it fits into 8 bits.
        This gives the branch instruction the range of -510 to 512.
        
        \begin{figure}[h]
          \begin{tabular}{c c c c}
            Bits & Mnemonic & Meaning            & Status register mapping\\
            \hline
            0000 & EQ & Equal                    & Z = 1\\
            0001 & NE & Not equal                & Z = 0\\
            0010 & CS & Carry set                & C = 1\\
            0010 & HS & Higher/same (unsigned)   & Same as CS\\
            0011 & CC & Carry clear              & C = 0\\
            0011 & LO & Lower (unsigned)         & Same as CC\\
            0100 & MI & Negative                 & N = 1\\
            0101 & PL & Positive                 & N = 0\\
            0110 & VS & Overflow set             & V = 1\\
            0111 & VC & Overflow clear           & V = 0\\
            1000 & HI & Higher (unsigned)        & C = 1 and Z = 0\\
            1001 & LS & Lower or same (unsigned) & C = 0 or Z = 1\\
            1010 & GE & Greater than or equal    & N = V\\
            1011 & LT & Less than                & N $\ne$ V\\
            1100 & GT & Greater than             & N = V and Z = 0\\
            1101 & LE & Less than or equal       & N $\ne$ V or Z = 1\\
            1110 & AL & Always                   & Any, this is the default\\
            1111 & NV & \emph{BNV extension space} & Do not use\\
          \end{tabular}
          \caption{Condition code bit mappings and meanings}
        \end{figure}
        
      
      \subsection*{Operation}
      
      \texttt{if cond is satisfied, \emph{pc} $\gets$ \emph{pc} + offset}
      
      \newpage

    \section{Addition}
      \subsection{ADD(1) --- Add two registers and a 4 bit immediate}
        \begin{center}\begin{bytefield}{16}
          \bitheader[b]{0, 3, 4, 7, 8, 11, 12, 13, 15}\\
          \bitbox[ltb]{1}{0} & \bitbox[tb]{1}{0} & \bitbox[tb]{1}{1} & 
          \bitbox{1}{0} &
          \bitbox{4}{\emph{Rd}} &
          \bitbox{4}{immed} &
          \bitbox{4}{\emph{Rs}}
        \end{bytefield}\end{center}
         Add two registers together, and then add an unsigned 4 bit integer,
         and store the result in the first register.
         \subsubsection*{Syntax}
           \texttt{ADD <\emph{Rd}>, <\emph{Rs}>, \#immediate}
         \subsubsection*{Operation}
           \texttt{\emph{Rd} $\gets$ \emph{Rd} + \emph{Rs} +
             SignExtend(immediate)}
         
      \subsection{ADD(2) --- Add register and an 8 bit immediate}
        \begin{center}\begin{bytefield}{16}
          \bitheader[b]{0, 7, 8, 11, 12, 13, 15}\\
          \bitbox[ltb]{1}{0} & \bitbox[tb]{1}{0} & \bitbox[tb]{1}{1} & 
          \bitbox{1}{1} &
          \bitbox{4}{\emph{Rd}} &
          \bitbox{8}{immed}
        \end{bytefield}\end{center}
         Add the contents of a register and an unsigned 8 bit integer and
         store the result back into the register.
         \subsubsection*{Syntax}
           \texttt{ADD <\emph{Rd}>, \#immediate}
         \subsubsection*{Operation}
           \texttt{\emph{Rd} $\gets$ \emph{Rd} + SignExtend(immediate)}
    
      \newpage
      
    \section{Subtraction}
      \subsection{SUB(1) --- Subtracts two registers and a 4 bit immediate}
        \begin{center}\begin{bytefield}{16}
          \bitheader[b]{0, 3, 4, 7, 8, 11, 12, 13, 15}\\
          \bitbox[ltb]{1}{0} & \bitbox[tb]{1}{1} & \bitbox[tb]{1}{0} & 
          \bitbox{1}{0} &
          \bitbox{4}{\emph{Rd}} &
          \bitbox{4}{immed} &
          \bitbox{4}{\emph{Rs}}
        \end{bytefield}\end{center}
         Substract two registers from each other, and then subtract an unsigned
         4 bit integer, and store the result in the first register.
         \subsubsection*{Syntax}
           \texttt{SUB <\emph{Rd}>, <\emph{Rs}>, \#immediate}
         \subsubsection*{Operation}
           \texttt{\emph{Rd} $\gets$ \emph{Rd} - \emph{Rs} - 
             SignExtend(immediate)}
         
      \subsection{SUB(2) --- Subtracts register and an 8 bit immediate}
        \begin{center}\begin{bytefield}{16}
          \bitheader[b]{0, 7, 8, 11, 12, 13, 15}\\
          \bitbox[ltb]{1}{0} & \bitbox[tb]{1}{1} & \bitbox[tb]{1}{0} &
          \bitbox{1}{1} &
          \bitbox{4}{\emph{Rd}} &
          \bitbox{8}{immed}
        \end{bytefield}\end{center}
         Subtracts the contents of a register and an unsigned 8 bit integer and
         store the result back into the register.
         \subsubsection*{Syntax}
           \texttt{SUB <\emph{Rd}>, \#immediate}
         \subsubsection*{Operation}
           \texttt{\emph{Rd} $\gets$ \emph{Rd} - SignExtend(immediate)}
    
      \newpage   
    
    \section{Comparison}
      \subsection{CMP(1) --- Compare register to an 8 bit signed immediate}
        \begin{center}\begin{bytefield}{16}
	  \bitheader[b]{0, 7, 8, 11, 12, 13, 15}\\
	  \bitbox[ltb]{1}{0} & \bitbox[tb]{1}{1} & \bitbox[tb]{1}{1} &
          \bitbox{1}{0} &
          \bitbox{4}{\emph{Rs}} &
          \bitbox{8}{immed}
        \end{bytefield}\end{center}
        
        Compare the contents of a register to an 8 bit signed immediate value,
        and update the status flags reflecting the results.
        \subsubsection*{Syntax}
          \texttt{CMP <\emph{Rs}>, \#immediate}
        \subsubsection*{Operation}
          \begin{texttt}
            alu\_out $\gets$ \emph{Rs} - SignExtend(immediate)\\
            if (alu\_out) is negative then \emph{sr}[N] $\gets$ 1 else 0\\
            if (alu\_out) = 0 then \emph{sr}[Z] $\gets$ 1 else 0\\
            if (SignExtend(immediate) $\ge$ \emph{Rs}) then \emph{sr}[C] $\gets$ 1 else 0\\
            if OverflowFrom(\emph{Rs} - SignExtend(immediate)) then \emph{sr}[V] $\gets$ 1 else 0
          \end{texttt}

      \subsection{CMP(2) --- Compare two registers from either banks}
        \begin{center}\begin{bytefield}{16}
	  \bitheader[b]{0, 3, 4, 5, 6, 7, 8, 11, 12, 13, 15}\\
	  \bitbox[ltb]{1}{0} & \bitbox[tb]{1}{1} & \bitbox[tb]{1}{1} &
          \bitbox{1}{1} &
          \bitbox{4}{\emph{Rn}} &
          \bitbox{1}{0} &
          \bitbox{1}{0} &
          \bitbox{1}{$B^{n}$} &
          \bitbox{1}{$B^{x}$} &
          \bitbox{4}{\emph{Rx}}
        \end{bytefield}\end{center}
        
        Compare the contents of two registers, either from either of the
        register banks, and update the status flags reflecting the results.
        The \emph{$B^{n}$} and \emph{$B^{x}$} bits are set to 1 if the corresponding
        \emph{Rn} or \emph{Rx} are to be read from the alternative bank.
        \subsubsection*{Syntax}
          \texttt{CMP <\emph{\{A\}Rn}>, <\emph{\{A\}Rx}>}
        \subsubsection*{Operation}
          \begin{texttt}
            alu\_out $\gets$ \emph{Rn} - \emph{Rx}\\
            if (\emph{Rn} - \emph{Rx}) is negative then \emph{sr}[N] $\gets$ 1 else 0\\
            if (\emph{Rn} - \emph{Rx}) = 0 then \emph{sr}[Z] $\gets$ 1 else 0\\
            if (\emph{Rx} $\ge$ \emph{Rn}) then \emph{sr}[C] $\gets$ 1 else 0\\
            if OverflowFrom(\emph{Rn} - \emph{Rx}) then \emph{sr}[V] $\gets$ 1 else 0
          \end{texttt}

      \subsection{TST --- Test if a bit is set in a register}
        \begin{center}\begin{bytefield}{16}
          \bitheader[b]{0, 4, 5, 6, 7, 8, 11, 12, 13, 15}\\
          \bitbox[ltb]{1}{0} & \bitbox[tb]{1}{1} & \bitbox[tb]{1}{1} &
          \bitbox{1}{1} &
          \bitbox{4}{\emph{Rn}} &
          \bitbox{1}{1} &
          \bitbox{1}{0} &
          \bitbox{1}{$B^{n}$} &
          \bitbox{5}{immed}
        \end{bytefield}\end{center}
        
        Test to see if a specific bit is set in a register, which can be from
        either bank, and update the status flags reflecting the results.  The
        instruction stores which bit to test in a 5 bit unsigned immediate.
        If the \emph{$B^{n}$} bit is set, then the register compared against is
        from the alternative bank.
        \subsubsection*{Syntax}
          \texttt{TST <\emph{\{A\}Rs}>, \#immediate}
        \subsubsection*{Operation}
          \begin{texttt}
            alu\_out $\gets$ \emph{Rn} AND (1 $\ll$ immediate)\\
            \emph{sr}[N] $\gets$ alu\_out[31]\\
            if alu\_out = 0 then \emph{sr}[Z] $\gets$ 1 else 0
          \end{texttt}
      \newpage
     
    \section{Moving Registers}
      \subsection{MOV --- Move contents of registers, with optional byte swapping}
        \begin{center}\begin{bytefield}{16}
          \bitheader[b]{0, 3, 4, 5, 6, 7, 8, 11, 12, 13, 15}\\
          \bitbox[ltb]{1}{1} & \bitbox[tb]{1}{0} & \bitbox[tb]{1}{0} &
          \bitbox{1}{0} &
          \bitbox{4}{\emph{Rd}} &
          \bitbox{1}{B} &
          \bitbox{1}{W} &
          \bitbox{1}{$B^{d}$} &
          \bitbox{1}{$B^{s}$} &
          \bitbox{4}{\emph{Rs}}
        \end{bytefield}\end{center}
        
        Moves the contents of one register to another, overwriting the
        destiniation's current value.  It can optionally read or write to
        the alternative bank by appropriately setting \emph{$B^{d}$} or
        \emph{$B^{s}$}.  It can also perform some data manipulation while
        the move happens.  The \emph{B} bit will cause all the bytes in the
        register to be swapped with their neighbour.  The \emph{W} bit will
        swap the two halves of the register.  Combining both bits will give
        you a endian swap.
        
        \subsubsection*{Syntax}
          \texttt{MOV\{B\}\{W\} <\emph{\{A\}Rd}>, <\emph{\{A\}Rs}>}
          
        \subsubsection*{Operation}
          alu\_out $\gets$ \emph{Rs}\\
          if B bit, alu\_out $\gets$ (alu\_out \& 0xff00ff00) $\gg$ 8 ORR
                                     (alu\_out \& 0x00ff00ff) $\ll$ 8\\
          if W bit, alu\_out $\gets$ (alu\_out $\ll$ 16) ORR
                                     (alu\_out $\gg$ 16)\\
          \emph{Rd} $\gets$ alu\_out
      
        \subsection{LDI --- Load 12 bit signed immediate into IR/R13}
          \begin{center}\begin{bytefield}{16}
            \bitheader[b]{0, 11, 12, 13, 15}\\
            \bitbox[ltb]{1}{1} & \bitbox[tb]{1}{0} & \bitbox[tb]{1}{0} &
            \bitbox{1}{1} &
            \bitbox{12}{immed}
          \end{bytefield}\end{center}
          
          Moves a 12 bit signed immediate value into \emph{R13}, the
          Immediate Register, or \emph{ir} for short.  This allows values from
          -2048 to 2047 to be easily obtained.
          
          \subsubsection*{Syntax}
            \texttt{LDI \#immediate}
            
          \subsubsection*{Operation}
            \emph{ir} $\gets$ SignExtend(immediate)
      
      \newpage
      
    \section{Bit Shifting and Rotating}
      By immediate constant:\\*
      \begin{center}\begin{bytefield}{16}
        \bitheader[b]{0, 4, 5, 6, 7, 8, 11, 12, 13, 15}\\
        \bitbox[ltb]{1}{1} & \bitbox[tb]{1}{0} & \bitbox[tb]{1}{1} &
        \bitbox{1}{A} &
        \bitbox{4}{\emph{Rd}} &
        \bitbox{1}{D} &
        \bitbox{1}{R} &
        \bitbox{1}{0} &
        \bitbox{5}{immed}
      \end{bytefield}\end{center}
      By value in register:\\*
      \begin{center}\begin{bytefield}{16}
        \bitheader[b]{0, 3, 4, 5, 6, 7, 8, 11, 12, 13, 15}\\
        \bitbox[ltb]{1}{1} & \bitbox[tb]{1}{0} & \bitbox[tb]{1}{1} &
        \bitbox{1}{A} &
        \bitbox{4}{\emph{Rd}} &
        \bitbox{1}{D} &
        \bitbox{1}{R} &
        \bitbox{1}{1} &
        \bitbox{1}{0} &
        \bitbox{4}{\emph{Rs}}
      \end{bytefield}\end{center}    
            
      Precisely what this instruction does is controlled by the bits
      \emph{A}, \emph{D} and \emph{R}.  Setting them in different ways allows
      you to recreate ARM-like instructions such as LSL, LSR, ROL, ROR, ASR,
      etc:
      \\
      \\
      \begin{tabular}{c l}
        Bit & Meaning\\
        \hline
        A & Style, 1 $\Rightarrow$ arithmetic, 0 $\Rightarrow$ logical\\
        D & Direction, 1 $\Rightarrow$ right, 0 $\Rightarrow$ left\\
        R & Type, 1 $\Rightarrow$ rotate, 0 $\Rightarrow$ shift
      \end{tabular}
      \\
      \\
      The immediate is a 5 bit unsigned value describing a number of
      positions to shift or roll, between 0 and 31.
      
      \subsection*{Syntax}
        Due to this instruction's flexibility, several mnemonics are
        suggested for it, depending on the combinations of option bits that
        are set.  These are:
        \\
        \\
        \begin{tabular}{c l}
          Mnemonic & Purpose\\
          \hline
          LSL & Logical Shift Left\\
          LSR & Logical Shift Right\\
          ASL & Arithmetic Shift Left\\
          ASR & Arithmetic Shift Right\\
          ROL & Rotate Left\\
          ROR & Rotate Right
        \end{tabular}
        \\
        All of these can be used in either the immediate or register forms:
        \\
        \\*
        \texttt{LSL <\emph{Rd}>, \#immediate\\
        ROR <\emph{Rd}>, <\emph{Rs}>}
      
      \subsection*{Operation}
        shift\_value $\gets$ \#immediate if immediate form, \emph{Rs} otherwise\\
        \emph{Rd} $\gets$ \emph{Rd} OPERATION shift\_value
        
      \newpage
      
      
    \section{Bit Manipulation - BIT}
      The BIT instruction is one of the most expressive, and most assemblers
      should provide macros that make it more usable, much like the bit shift
      and rotate instruction discussed in the previous section does.
      
      By immediate value:\\*
      \begin{center}\begin{bytefield}{16}
        \bitheader[b]{0, 4, 5, 6, 7, 8, 11, 12, 13, 15}\\
        \bitbox[ltb]{1}{1} & \bitbox[tb]{1}{1} & \bitbox[tb]{1}{0} &
        \bitbox{1}{N} &
        \bitbox{4}{\emph{Rd}} &
        \bitbox{2}{OP} &
        \bitbox{1}{1} &
        \bitbox{5}{immed}
      \end{bytefield}\end{center}

      By value in register:\\*
      \begin{center}\begin{bytefield}{16}
        \bitheader[b]{0, 3, 4, 5, 6, 7, 8, 11, 12, 13, 15}\\
        \bitbox[ltb]{1}{1} & \bitbox[tb]{1}{1} & \bitbox[tb]{1}{0} &
        \bitbox{1}{N} &
        \bitbox{4}{\emph{Rd}} &
        \bitbox{2}{OP} &
        \bitbox{1}{0} &
        \bitbox{1}{0} &
        \bitbox{4}{\emph{Rs}}
      \end{bytefield}\end{center}
      
      If the N bit is set to 1, the value of the operand is negated before
      the operation is performed.  It does not alter the value in the register
      if the register form is used.
      
      \begin{center}\begin{figure}[h]
        \begin{tabular}{c c l}
          Bits & Meaning & Operation\\
          \hline
          00 & NOT & \emph{Rd} $\gets$ NOT (operand)\\
          01 & AND & \emph{Rd} $\gets$ \emph{Rd} AND (operand)\\
          10 & OR  & \emph{Rd} $\gets$ \emph{Rd} OR (operand)\\
          11 & EOR & \emph{Rd} $\gets$ \emph{Rd} EOR (operand)
        \end{tabular}
        \caption{Operations performed by the BIT instruction}
      \end{figure}\end{center}

      Several ARM-like instructions can be constructed using this opcode past the obvious
      \emph{NOT, AND, OR,} and \emph{EOR}.  Suggested ones are:

      \begin{tabular}{c l l }
        Mnemonic & Meaning & Bits set\\
	\hline
	NOT & Negate register & N $\Leftarrow$ 0, OP $\Leftarrow$ 00, both registers the same\\
	MVN & Move negative value into register & N $\Leftarrow$ 0, OP $\Leftarrow$ 00, registers differ\\
	AND & Bitwise AND & N $\Leftarrow$ 0, OP $\Leftarrow$ 01\\
	ORR & Bitwise OR & N $\Leftarrow$ 0, OP $\Leftarrow$ 10\\
	BIS & Set a bit & N $\Leftarrow$ 0, OP $\Leftarrow$ 10\\
	EOR & Bitwise EOR & N $\Leftarrow$ 0, OP $\Leftarrow$ 11\\
	BIC & Clear a bit & N $\Leftarrow$ 1, OP $\Leftarrow$ 01
      \end{tabular}
  
    \section{Memory Access - MEM}
      The MEM instruction allows for the transfer of data between registers and
      memory.  It also provides functionality for adjusting the address register
      after the access.  It operates on two registers: one containing the address
      to read from or write to, and another that stores the result or the data to
      write.  All memory accesses must be aligned to their size.

      \begin{center}\begin{bytefield}{16}
        \bitheader[b]{0, 3, 4, 5, 6, 7, 8, 11, 12, 13, 15}\\
	\bitbox[ltb]{1}{1} & \bitbox[tb]{1}{1} & \bitbox[tb]{1}{1} &
	\bitbox{1}{L} &
	\bitbox{4}{\emph{Rv}} &
	\bitbox{2}{S} &
	\bitbox{2}{W} &
	\bitbox{4}{\emph{Ra}}
      \end{bytefield}\end{center}

      \emph{Rv} is the register containing the value to write, or the register into
      which to load.  \emph{Ra} is the register containing the address to act on.
      \\
      \begin{tabular}{c l}
         Bit & Meaning\\
         \hline
	 L & 1 $\Rightarrow$ load, 0 $\Rightarrow$ store\\
	 S & 00 $\Rightarrow$ 8 bit access\\
	   & 01 $\Rightarrow$ 32 bit access\\
	   & 10 $\Rightarrow$ 16 bit access, into high half of register\\
	   & 11 $\Rightarrow$ 16 bit access, into low half of register\\
	 W & 00 $\Rightarrow$ No action\\
	   & 01 $\Rightarrow$ UNDEFINED\\
	   & 10 $\Rightarrow$ Decrease address register by access size after access\\
	   & 11 $\Rightarrow$ Increase address register by access size after access
      \end{tabular}

      \subsection*{Syntax}
        \texttt{LDR[\emph{S}][\emph{W}] \emph{Rv}, \emph{Ra}\\
	        STR[\emph{S}][\emph{W}] \emph{Rv}, \emph{Ra}}\\
	
	\emph{S} may be one of \emph{B, H, L,} or omitted, meaning 8 bit, high 16 bit,
	low 16 bit, and 32 bit respectively.

	\emph{W} may be one of \emph{D, I,} or omitted, meaning decrease after access,
	increase after access, or no write back, respectively.
    

  \chapter{Memory Layout}
  
  \chapter{System Controller}
    \section{Chip-select information table}
    \section{Interrupt controller}
    \section{Programmable interrupt timer}
    \section{Trivial console serial port}
    
  \chapter{MEOW C ABI}
    \section{Introduction}
      Given MEOW is partially inspired by ARM, its ABI is partially inspired by
      ARM's ABI, called APCS.

      The MEOW ABI (MABI) only covers the active register set. If an operation
      switches the register banks (E.g. taking an interrupt) then the other
      register set has its own set of argument, scratch, etc registers.
    
    \section{Register use}
      MABI defines uses for all of MEOW's registers.  The registers are grouped
      into three main flavors:
      \begin{enumerate}
        \item four argument registers which can also be used as scratch 
          registers, or as caller-saved register variables;
        \item six callee-saved registers, typically used for register variables;
        \item and 6 registers that have dedicated use, at least some of
          the time.
      \end{enumerate}
      
      \begin{tabular}{c c l}
        Name & Number & MABI Purpose\\
        \hline
        a1 & 0 & argument 1 / integer return / scratch register\\
        a2 & 1 & argument 2 / scratch register\\
        a3 & 2 & argument 3 / scratch register\\
        a4 & 3 & argument 4 / scratch register\\
        \\
        v1 & 4 & variable register\\
        v2 & 5 & variable register\\
        v3 & 6 & variable register\\
        v4 & 7 & variable register\\
        v5 & 8 & variable register\\
        v6 & 9 & variable register\\
        \\
        at & 10 & compiler and assembler temporary / scratch register\\
        sp & 11 & stack pointer\\
        lr & 12 & link register / scratch register\\
        ir & 13 & immediate register / scratch register\\
        sr & 14 & CPU status register\\
        pc & 15 & program counter
      \end{tabular}
          
    \section{Floating Point registers}
      MEOW currently has no support for floating point in hardware.  It is
      suggested that float-point be implemented using function calls, but
      this has not yet been specified.
    
    \section{Stack}
      A full-decending stack is provided in sp(r11).  It is assumed that it
      provides enough storage for the running program: no checks are made
      as to if it has over-flowed.

      It is suggested that function local variables be allocated on the stack
      and used through it as to provide re-entrancy.
    
    \section{Function calls}
      \subsection{Function call arrival}
        A function can expect the following to be true when it is called:
        \begin{itemize}
          \item \emph{pc} shall contain the address of the function's
            first instruction;
          \item \emph{sp} shall point to a fill-decending stack ready for use;
          \item \emph{lr} shall contain the address that the function should
            return to;
          \item function arguments will be available in the strucure described
            in the next section.
        \end{itemize}

      \subsection{Function call arguments}
        The first four arguments are placed in a1 to a4.  Any additional
        arguments are pushed onto the stack in such an order that in increasing
        memory locations they are ordered as they are mentioned in the source
        program.  Thus, they are pushed in reverse order onto the stack.

        A callee is at liberty to corrupt its arguments, regardless of how they
        are delivered.

        All arguments are passed as 32 bit values, although they may only
        contain 16 or 8 bits of meaningful data, as well as 32 bits.
        
      \subsection{Function call return}
          When a function returns to its caller by placing \emph{lr} into
          \emph{pc}, the following must be true:
          \begin{itemize}
            \item \emph{ap} and \emph{v1} - \emph{v6} must have the same values
              as they did the instant before the function call;

            \item if the function returns a result, it shall be returned in
              \emph{a1}.
          \end{itemize}

        \subsubsection{Non-simple return types}
          A non-simple return value is one that will not fit into a word, such
          as a structure.  XXX TODO: We havn't decided how to do these yet.

\end{document}

